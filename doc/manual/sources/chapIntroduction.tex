\chapter{Introduction}
\label{secInrto}

Arduino\footnote{See www.arduino.cc} is a popular open source and open
hardware micro controller platform for various purposes, mainly located in leisure
time projects. Arduino comes along with a simple to use integrated
development en\-vi\-ron\-ment, which contains the complete tool chain to
write source code, to browse through samples and libraries, to compile and
link the software and to upload it to the board and flash it. The
\rtos{} project adds the programming paradigm of a real time operating
system to the Arduino world.

Real time operating systems, or RTOS, strongly simplify the implementation
of technical applications which typically do things in a quite regular
way, like checking inputs and setting outputs accordingly every (fixed)
fraction of a second. For example, the temperature controller for a
heating installation could be designed this way. Temperature sensors,
which report the room temperatures are evaluated and the burner and maybe
some valves are controlled to yield the desired target temperature.
Furthermore, using a real time system the program could coincidentally and
regularly update a display to provide feedback -- at the same or any other
rate. Regular, time based programming can be done without the need of CPU
consuming waiting loops as used in the implementation of Arduino's library
functions \ident{delay} and \ident{delayMicroseconds}. Real time operating
systems characterize the professional use of micro controllers.

\rtos{} is a small real time operating system (RTOS) for the Arduino
environment. It is simple to use and fits well into the existing Arduino
code environment. It adds the concept of pseudo-parallel execution threads
to the sketches.
%These threads are basically preemptive but can be configured to implement
%deliberate, cooperative task switching, too. Events can be sent and
%received by tasks as a simple kind of communication.

The traditional Arduino sketch has two entry points; the function
\ident{setup}, which is the place to put the initialization code required
to run the sketch and function \ident{loop}, which is periodically
called. The frequency of looping is not deterministic but depends on
the execution time of the code inside the loop.

Using \rtos{}, the two mentioned functions continue to exist and continue
to have the same meaning. However, as part of the code initialization in
\ident{setup} you may define a number of tasks having individual
properties. The most relevant property of a task is a C code
function\footnote{The GNU C compiler is quite uncomplicated in mixing C
and C++ files. Although \rtos{} is written in C it's no matter do
implement task functions in C++ if only the general rules of combining C
and C++ and the considerations about using library functions (particularly
\ident{new}) in a multi-tasking environment are obeyed. Non-static class
member functions are obviously no candidates for a task function.}, which
becomes the so called task function. Once entering the traditional Arduino
loop, all of these task functions are executed in parallel to one another
and to the repeated execution of function \ident{loop}. We say,
\ident{loop} becomes the idle task of the RTOS.

A characteristic of \rtos{} is that the behavior of a task is not fully
predetermined at compile time. \rtos{} supports regular, time-controlled
tasks as well as purely event controlled ones, where events can be
broadcasted or behave as mutex or semaphore. Tasks can be preemptive or
interact cooperatively. Task scheduling can be done using time slices and
a round robin pattern. Moreover, many of these modes can be mixed. A task
is not per se regular, its implementing code decides what happens and this
can be decided context or situation dependent. This flexibility is
achieved by the basic idea of having an event controlled scheduler, where
typical RTOS use cases are supported by providing according events, e.g.
absolute-point-in-time-reached. If the task's code decides to always wait
for the same absolute-point-in-time-reached event, then it becomes a
regular task. However, situation dependent the same task could decide to
wait for an application sent event -- and give up its regular behavior. In
many RTOS implementations the basic characteristic of a task is determined
at compile time, in \rtos{} this is done partly at compile time and partly
at runtime.

\rtos{} is provided as a single source code file which is compiled together
with your other code, which now becomes an \rtos{} application. In the most
simple case, if you do not define any task, your application will strongly
resemble a traditional sketch: You implement your \ident{setup} and your
\ident{loop} function; the former will be run once at the beginning and
the latter repeatedly.

\rtos{} on its own can't be compiled, there need to be an application.
\rtos{} is organized as a package which combines the \rtos{} source file
with some sample applications which are the test cases at the same time.
The source code of each sample application is held in a separate folder,
named tc\textless nn\textgreater. Any of these can be selected for
compilation. You may add more folders, holding the source code of your
\rtos{} applications. A starting point of your application folder can be a
copy of any of the folders tc\textless nn\textgreater. The compilation
always is the same. Run the makefile, where the name of the folder (which
doesn't need to be tc\textless nn\textgreater) is an option on the
command line. See section \ref{secMakefile} for more.

This document introduces the basic concept of \rtos{} and gives an
overview of its features and limitations:

Chapter \ref{secHowDoesRTuinOSWork} explains the basic principles of
operation. Some core considerations of the implementation are highlighted,
but the relevant documentation of the implementation is the code itself.
It is commented using doxygen\footnote{See http://www.doxygen.org} tags;
the compiled doxygen documentation is part of this project. It contains
only the documentation of the global objects of \rtos; to fully understand
the implementation you will have to inspect the source code itself, please
refer to \refRTOSC, \refRTOSH, \refRTOSConfigTemplateH.


Chapter \ref{secAPI} lists and documents all elements of
\rtos' API.

Chapter \ref{secHowToWriteApp} explains how to write a well-working
\rtos-application. The chapter starts with a short recipe, which
guarantees soon success. Here is where you may start reading if you are
already familiar with the concept of an RTOS.

The manual closes with chapter \ref{secOutlook}, which gives an overview
of possible improvements and still missing and maybe later released
features.

